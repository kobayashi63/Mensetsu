\documentclass[dvipdfmx, xcolor=svgnames]{beamer}
%\documentclass[xcolor=svgnames]{beamer}

\usetheme{Madrid}
\setbeamertemplate{navigation symbols}{}
\setbeamertemplate{footline}[frame number]
\setbeamertemplate{footline}[section]
\usepackage{graphicx}
\usepackage{amsmath}
\usepackage{amssymb}
\usepackage{txfonts}
\usepackage{tabularx}
\usepackage{color}
%\mathversion{bold}
\renewcommand{\familydefault}{\sfdefault}
\renewcommand{\kanjifamilydefault}{\gtdefault}
\setbeamerfont{title}{size=\large,series=\bfseries}
\setbeamerfont{frametitle}{size=\large,series=\bfseries}
%\setbeamerfont{framesubtitle}{size=\small,series=\bfseries}
\setbeamercolor{title}{fg=white, bg=Navy}
\setbeamertemplate{frametitle}[default][left]
\setbeamercolor{frametitle}{fg=white, bg=Navy}
%\setbeamercolor{framesubtitle}{fg=white, bg=black} %%%%%%%%%% 反映されないよ
\usefonttheme{professionalfonts}

\usepackage[deluxe]{otf}
\usepackage[noalphabet]{pxchfon}
\setboldgothicfont{HaranoAjiGothic-Medium.otf}

\theoremstyle{plain}
%\setbeamertemplate{theorems}
%\newtheorem{main}{Main Result}
\newtheorem{thm}{Theorem}[section]
\newtheorem{cor}{Corollary}[section]
\newtheorem{prop}{Proposition}[section]
\newtheorem{lemm}{Lemma}[section]
\theoremstyle{definition}
\newtheorem*{defi}{Definition}
%\newtheorem{fact}{Fact}[section]
\newtheorem{question}{Question}
\newtheorem{exam}{Example}[section]
\newtheorem{assumption}{Assumption}[section]
%\numberwithin{equation}{section}
\theoremstyle{remark}
\newtheorem{remark}{Remark}[section]


\usepackage{amsmath,amssymb,amsthm,amscd,ascmac,mathrsfs,bm,type1cm,multirow,array,enumerate,mathrsfs,tabularx}
\usepackage{wrapfig,float}
% \usepackage{graphicx}
\usepackage{color}
\usepackage{arydshln} % 行列 c:c 
% \usepackage[dvipdfmx]{graphicx}

\renewcommand{\arraystretch}{1.1}

\newcounter{main} % カウンタの宣言
\setcounter{main}{0} % カウンタの初期化
\resetcounteronoverlays{main}
\newcommand{\useMain}[1][]{\refstepcounter{main}{#1}Main Result \,{\themain}}


\newcounter{Prop} % カウンタの宣言
\setcounter{Prop}{0} % カウンタの初期化
\resetcounteronoverlays{Prop}
\newcommand{\useProp}[1][]{\refstepcounter{Prop}{#1}Proposition \,{\theProp}}



\def\rnum#1{\expandafter{\romannumeral #1}} 
\def\Rnum#1{\uppercase\expandafter{\romannumeral #1}} 
\def\hsymb#1{\mbox{\strut\rlap{\smash{\Huge$#1$}}\quad}} % 行列の省略記号*

\newcommand{\BigFig}[1]{\parbox{12pt}{\Huge #1}}
\newcommand{\BigZero}{\BigFig{0}} 


\setcounter{tocdepth}{1}


\title{研究内容の紹介(20分)}
\author{東京理科大学理学研究科 \quad
        小林 穂乃香}
%\institute{小池研究室}
\date{山口大学面接\, 於\\
      2022年9月5日}

\begin{document}
%%%%% page 1 (title page) %%%%%
\frame{\titlepage}




\section{擬リーマン多様体とは}

\frame{
\frametitle{\insertsection}
\hspace{10pt}$g$\hspace{10pt} : \(M\)上の対称$(0,2)$テンソル場\\
\hspace{5pt}$X,\, Y\, \in TM$ \vspace{10pt}\\
\underline{$(M, g)$ : リーマン多様体}\vspace{-5pt}
\begin{table}
\begin{tabular}{lll}
\hspace{-3pt}  
$g$ : $M$上のリーマン計量 
\hspace{10pt}
&
\uncover<1->{\hspace{4pt} i. e.}
&
\uncover<1->{$g(X,X)\geq0$ {\it \, for all $X$, and}}
\\
 \hspace{48pt}\(\cdots \)正定値
&
&
\uncover<1->{$g(X,X)=0 \Leftrightarrow X=0$}
\end{tabular}
\vspace{6pt}
\end{table}
\uncover<1->{
\underline{$(M, g)$ : 擬リーマン多様体}\vspace{-5pt}
\begin{table}
  \hspace{20pt}
\begin{tabular}{lll}
  \hspace{-24pt}
$g$ : $M$上の擬リーマン計量 
&
\uncover<1->{\hspace{8pt}i. e. }
&
\uncover<1->{
  \hspace{-9pt}
$g(X,Y)=0$\, {\it for all}\, $Y\in TM$ 
}
\\
\uncover<1->{\hspace{28pt}\(\cdots \) 非退化}
&
&
\uncover<1->{\hspace{30pt} $\Rightarrow X=0$}
\end{tabular}
\end{table}

\uncover<2->{
  \begin{defi}{}
    \begin{table}
    \begin{center}
    \begin{tabular}{lllcl}
    $X$ : \textcolor{red}{{\it spacelike}} & 
    $\overset{\mathrm{def}}{\Leftrightarrow}$ &
    $\langle X, X\rangle >0$ &
    or &
    $X=0$
    \\
    $X$ : \textcolor{red}{{\it timelike}} & 
    $\overset{\mathrm{def}}{\Leftrightarrow}$ & 
    $\langle X, X\rangle <0$ & 
    \\
    $X$ : \textcolor{red}{{\it null}} & 
    $\overset{\mathrm{def}}{\Leftrightarrow}$ &
    $\langle X, X\rangle =0$ &
    and &
    $X\neq 0$
    \end{tabular}
    \end{center}
    \vspace{-4pt}
    \end{table}
    \end{defi}
}
}}






\frame{
  \frametitle{\insertsection}
${\bf x}=(x_1,\ldots ,x_{m+1})$, \, ${\bf y}=(y_1,\ldots,y_{m+1})$ \, $\in \mathbb{E}^{m+1}_{\textcolor<2->{red}{s}}$
\hspace{5pt} 
\uncover<2->{
  \begin{minipage}{0.3\textwidth}
    {\small 
    \(
      \left\{
    \begin{array}{ll}
    s=0 \, \text{: リーマン多様体}\\
    s=1 \, \text{: ローレンツ多様体}
    \end{array}
      \right.
    \)
    }
  \end{minipage}
}
\vspace{5pt}

\hspace{20pt}
\(\langle {\bf x},{\bf y}\rangle :=\textcolor<2->{red}{-}\sum_{i=1}^{\textcolor<2->{red}{s}}x_iy_i +\sum_{j=\textcolor<2->{red}{s+1}}^{m+1}x_jy_j\)
\hspace{20pt}
\uncover<2->{\textcolor{red}{$s$ : 指数}}
\vspace{0pt}
\begin{table}
\hspace{0pt}
\begin{tabular}{rcl}
\hspace{-15pt}
\begin{minipage}[l]{0.4\textwidth}
\uncover<3->{\textcolor{black}{\underline{リーマン多様体}}}
\hspace{-3pt}
\vspace{5pt}
\end{minipage}
&
\hspace{-3pt}
&
\hspace{-20pt}
\begin{minipage}[l]{0.4\textwidth}
%\textcolor{black}{
\uncover<3->{\underline{擬リーマン多様体}}
%}
\vspace{5pt}
\end{minipage}
\\
\hspace{-15pt}
\begin{minipage}[l]{0.4\textwidth}
\shortstack{
%\textcolor{black}{$\mathbb{S}^2_1$ : pseudo-sphere in $\mathbb{R}^3_1$}
\uncover<3->{${\small \mathbb{S}^2:=\{ {\bf x}\in\mathbb{E}^3\,} {\large \vert} {\small \, \langle {\bf x},{\bf x}\rangle =1 \} }$}\\
\hspace{5pt}\uncover<4->{${\small \mathbb{H}^2:=\{ {\bf x}\in\mathbb{E}^3_1\,} {\large \vert} {\small \, \langle {\bf x},{\bf x}\rangle =-1 \} }$}
}
\end{minipage}
\vspace{7pt}
&
&
\hspace{-20pt}
\begin{minipage}[l]{0.4\textwidth}
\shortstack{
%\textcolor{black}{$\mathbb{S}^2_1$ : pseudo-sphere in $\mathbb{R}^3_1$}
\uncover<3->{${\small \mathbb{S}^2_1:=\{ {\bf x}\in\mathbb{E}^3_1\,} {\large \vert} {\small \, \langle {\bf x},{\bf x}\rangle =1 \} }$}\\
\hspace{5pt}\uncover<4->{${\small \mathbb{H}^2_1:=\{ {\bf x}\in\mathbb{E}^3_2\,} {\large \vert} {\small \, \langle {\bf x},{\bf x}\rangle =-1 \} }$}
}
\end{minipage}
\\
\begin{minipage}[c]{0.4\textwidth}
\hspace{10pt}
\uncover<3->{
\scalebox{0.5}{%WinTpicVersion4.32a
{\unitlength 0.1in%
\begin{picture}(28.6000,26.5000)(4.2000,-30.0000)%
% STR 2 0 3 0 Black White  
% 4 1840 1777 1840 1790 4 2400 0 0
% O
\put(18.4000,-17.9000){\makebox(0,0)[rt]{O}}%
% VECTOR 2 0 3 0 Black White  
% 2 1850 3000 1850 350
% 
\special{pn 8}%
\special{pa 1850 3000}%
\special{pa 1850 350}%
\special{fp}%
\special{sh 1}%
\special{pa 1850 350}%
\special{pa 1830 417}%
\special{pa 1850 403}%
\special{pa 1870 417}%
\special{pa 1850 350}%
\special{fp}%
% VECTOR 2 0 3 0 Black White  
% 2 420 1780 3280 1780
% 
\special{pn 8}%
\special{pa 420 1780}%
\special{pa 3280 1780}%
\special{fp}%
\special{sh 1}%
\special{pa 3280 1780}%
\special{pa 3213 1760}%
\special{pa 3227 1780}%
\special{pa 3213 1800}%
\special{pa 3280 1780}%
\special{fp}%
% FUNC 2 0 3 0 Black White  
% 10 420 350 3280 3000 1850 1780 2950 1780 1850 680 420 350 3280 3000 50 2 0 3 1 0
% ///sin(t)^2+cos(t)^2///0///2pi
\special{pn 8}%
\special{pa 2950 1780}%
\special{pa 2950 1752}%
\special{pa 2949 1738}%
\special{pa 2949 1725}%
\special{pa 2946 1683}%
\special{pa 2944 1669}%
\special{pa 2943 1656}%
\special{pa 2939 1628}%
\special{pa 2937 1615}%
\special{pa 2933 1587}%
\special{pa 2930 1574}%
\special{pa 2928 1560}%
\special{pa 2925 1547}%
\special{pa 2922 1533}%
\special{pa 2919 1520}%
\special{pa 2915 1506}%
\special{pa 2912 1493}%
\special{pa 2908 1480}%
\special{pa 2904 1466}%
\special{pa 2892 1427}%
\special{pa 2887 1414}%
\special{pa 2883 1401}%
\special{pa 2868 1362}%
\special{pa 2862 1349}%
\special{pa 2857 1337}%
\special{pa 2851 1324}%
\special{pa 2845 1312}%
\special{pa 2839 1299}%
\special{pa 2833 1287}%
\special{pa 2827 1274}%
\special{pa 2820 1262}%
\special{pa 2814 1250}%
\special{pa 2779 1190}%
\special{pa 2771 1179}%
\special{pa 2764 1167}%
\special{pa 2748 1145}%
\special{pa 2740 1133}%
\special{pa 2732 1122}%
\special{pa 2723 1111}%
\special{pa 2715 1100}%
\special{pa 2706 1089}%
\special{pa 2697 1079}%
\special{pa 2689 1068}%
\special{pa 2680 1058}%
\special{pa 2670 1047}%
\special{pa 2652 1027}%
\special{pa 2642 1017}%
\special{pa 2633 1007}%
\special{pa 2623 997}%
\special{pa 2613 988}%
\special{pa 2603 978}%
\special{pa 2593 969}%
\special{pa 2583 959}%
\special{pa 2572 950}%
\special{pa 2562 941}%
\special{pa 2551 932}%
\special{pa 2540 924}%
\special{pa 2530 915}%
\special{pa 2519 907}%
\special{pa 2508 898}%
\special{pa 2496 890}%
\special{pa 2474 874}%
\special{pa 2462 866}%
\special{pa 2451 859}%
\special{pa 2439 851}%
\special{pa 2428 844}%
\special{pa 2368 809}%
\special{pa 2355 803}%
\special{pa 2331 791}%
\special{pa 2318 785}%
\special{pa 2306 779}%
\special{pa 2293 773}%
\special{pa 2280 768}%
\special{pa 2268 762}%
\special{pa 2229 747}%
\special{pa 2216 743}%
\special{pa 2203 738}%
\special{pa 2177 730}%
\special{pa 2163 726}%
\special{pa 2137 718}%
\special{pa 2123 715}%
\special{pa 2110 711}%
\special{pa 2097 708}%
\special{pa 2083 705}%
\special{pa 2070 702}%
\special{pa 2056 699}%
\special{pa 2042 697}%
\special{pa 2029 695}%
\special{pa 2015 692}%
\special{pa 2001 690}%
\special{pa 1988 689}%
\special{pa 1974 687}%
\special{pa 1960 686}%
\special{pa 1946 684}%
\special{pa 1933 683}%
\special{pa 1905 681}%
\special{pa 1891 681}%
\special{pa 1878 680}%
\special{pa 1822 680}%
\special{pa 1808 681}%
\special{pa 1795 681}%
\special{pa 1753 684}%
\special{pa 1739 686}%
\special{pa 1726 687}%
\special{pa 1698 691}%
\special{pa 1685 693}%
\special{pa 1657 697}%
\special{pa 1644 700}%
\special{pa 1630 702}%
\special{pa 1617 705}%
\special{pa 1603 708}%
\special{pa 1590 711}%
\special{pa 1576 715}%
\special{pa 1563 718}%
\special{pa 1550 722}%
\special{pa 1536 726}%
\special{pa 1497 738}%
\special{pa 1484 743}%
\special{pa 1471 747}%
\special{pa 1432 762}%
\special{pa 1419 768}%
\special{pa 1407 773}%
\special{pa 1394 779}%
\special{pa 1382 785}%
\special{pa 1369 791}%
\special{pa 1357 797}%
\special{pa 1344 803}%
\special{pa 1332 810}%
\special{pa 1320 816}%
\special{pa 1260 851}%
\special{pa 1249 859}%
\special{pa 1237 866}%
\special{pa 1215 882}%
\special{pa 1203 890}%
\special{pa 1192 898}%
\special{pa 1181 907}%
\special{pa 1170 915}%
\special{pa 1159 924}%
\special{pa 1149 933}%
\special{pa 1138 941}%
\special{pa 1128 950}%
\special{pa 1117 960}%
\special{pa 1097 978}%
\special{pa 1087 988}%
\special{pa 1077 997}%
\special{pa 1067 1007}%
\special{pa 1058 1017}%
\special{pa 1048 1027}%
\special{pa 1039 1037}%
\special{pa 1029 1047}%
\special{pa 1020 1058}%
\special{pa 1011 1068}%
\special{pa 1002 1079}%
\special{pa 994 1090}%
\special{pa 985 1100}%
\special{pa 977 1111}%
\special{pa 968 1122}%
\special{pa 960 1134}%
\special{pa 944 1156}%
\special{pa 936 1168}%
\special{pa 929 1179}%
\special{pa 921 1191}%
\special{pa 914 1202}%
\special{pa 879 1262}%
\special{pa 873 1275}%
\special{pa 861 1299}%
\special{pa 855 1312}%
\special{pa 849 1324}%
\special{pa 843 1337}%
\special{pa 838 1350}%
\special{pa 832 1362}%
\special{pa 817 1401}%
\special{pa 813 1414}%
\special{pa 808 1427}%
\special{pa 800 1453}%
\special{pa 796 1467}%
\special{pa 788 1493}%
\special{pa 785 1507}%
\special{pa 781 1520}%
\special{pa 778 1533}%
\special{pa 775 1547}%
\special{pa 772 1560}%
\special{pa 769 1574}%
\special{pa 767 1588}%
\special{pa 765 1601}%
\special{pa 762 1615}%
\special{pa 760 1629}%
\special{pa 759 1642}%
\special{pa 757 1656}%
\special{pa 756 1670}%
\special{pa 754 1684}%
\special{pa 753 1697}%
\special{pa 751 1725}%
\special{pa 751 1739}%
\special{pa 750 1752}%
\special{pa 750 1808}%
\special{pa 751 1822}%
\special{pa 751 1835}%
\special{pa 754 1877}%
\special{pa 756 1891}%
\special{pa 757 1904}%
\special{pa 761 1932}%
\special{pa 763 1945}%
\special{pa 767 1973}%
\special{pa 770 1986}%
\special{pa 772 2000}%
\special{pa 775 2013}%
\special{pa 778 2027}%
\special{pa 781 2040}%
\special{pa 785 2054}%
\special{pa 788 2067}%
\special{pa 792 2080}%
\special{pa 796 2094}%
\special{pa 808 2133}%
\special{pa 813 2146}%
\special{pa 817 2159}%
\special{pa 832 2198}%
\special{pa 838 2211}%
\special{pa 843 2223}%
\special{pa 849 2236}%
\special{pa 855 2248}%
\special{pa 861 2261}%
\special{pa 867 2273}%
\special{pa 873 2286}%
\special{pa 880 2298}%
\special{pa 886 2310}%
\special{pa 921 2370}%
\special{pa 929 2381}%
\special{pa 936 2393}%
\special{pa 952 2415}%
\special{pa 960 2427}%
\special{pa 968 2438}%
\special{pa 977 2449}%
\special{pa 985 2460}%
\special{pa 994 2471}%
\special{pa 1003 2481}%
\special{pa 1011 2492}%
\special{pa 1020 2502}%
\special{pa 1030 2513}%
\special{pa 1048 2533}%
\special{pa 1058 2543}%
\special{pa 1067 2553}%
\special{pa 1077 2563}%
\special{pa 1087 2572}%
\special{pa 1097 2582}%
\special{pa 1107 2591}%
\special{pa 1117 2601}%
\special{pa 1128 2610}%
\special{pa 1138 2619}%
\special{pa 1149 2628}%
\special{pa 1160 2636}%
\special{pa 1170 2645}%
\special{pa 1181 2653}%
\special{pa 1192 2662}%
\special{pa 1204 2670}%
\special{pa 1226 2686}%
\special{pa 1238 2694}%
\special{pa 1249 2701}%
\special{pa 1261 2709}%
\special{pa 1272 2716}%
\special{pa 1332 2751}%
\special{pa 1345 2757}%
\special{pa 1369 2769}%
\special{pa 1382 2775}%
\special{pa 1394 2781}%
\special{pa 1407 2787}%
\special{pa 1420 2792}%
\special{pa 1432 2798}%
\special{pa 1471 2813}%
\special{pa 1484 2817}%
\special{pa 1497 2822}%
\special{pa 1523 2830}%
\special{pa 1537 2834}%
\special{pa 1563 2842}%
\special{pa 1577 2845}%
\special{pa 1590 2849}%
\special{pa 1603 2852}%
\special{pa 1617 2855}%
\special{pa 1630 2858}%
\special{pa 1644 2861}%
\special{pa 1658 2863}%
\special{pa 1671 2865}%
\special{pa 1685 2868}%
\special{pa 1699 2870}%
\special{pa 1712 2871}%
\special{pa 1726 2873}%
\special{pa 1740 2874}%
\special{pa 1754 2876}%
\special{pa 1767 2877}%
\special{pa 1795 2879}%
\special{pa 1809 2879}%
\special{pa 1822 2880}%
\special{pa 1878 2880}%
\special{pa 1892 2879}%
\special{pa 1905 2879}%
\special{pa 1947 2876}%
\special{pa 1961 2874}%
\special{pa 1974 2873}%
\special{pa 2002 2869}%
\special{pa 2015 2867}%
\special{pa 2043 2863}%
\special{pa 2056 2860}%
\special{pa 2070 2858}%
\special{pa 2083 2855}%
\special{pa 2097 2852}%
\special{pa 2110 2849}%
\special{pa 2124 2845}%
\special{pa 2137 2842}%
\special{pa 2150 2838}%
\special{pa 2164 2834}%
\special{pa 2203 2822}%
\special{pa 2216 2817}%
\special{pa 2229 2813}%
\special{pa 2268 2798}%
\special{pa 2281 2792}%
\special{pa 2293 2787}%
\special{pa 2306 2781}%
\special{pa 2318 2775}%
\special{pa 2331 2769}%
\special{pa 2343 2763}%
\special{pa 2356 2757}%
\special{pa 2368 2750}%
\special{pa 2380 2744}%
\special{pa 2440 2709}%
\special{pa 2451 2701}%
\special{pa 2463 2694}%
\special{pa 2485 2678}%
\special{pa 2497 2670}%
\special{pa 2508 2662}%
\special{pa 2519 2653}%
\special{pa 2530 2645}%
\special{pa 2541 2636}%
\special{pa 2551 2627}%
\special{pa 2562 2619}%
\special{pa 2572 2610}%
\special{pa 2583 2600}%
\special{pa 2603 2582}%
\special{pa 2613 2572}%
\special{pa 2623 2563}%
\special{pa 2633 2553}%
\special{pa 2642 2543}%
\special{pa 2652 2533}%
\special{pa 2661 2523}%
\special{pa 2671 2513}%
\special{pa 2680 2502}%
\special{pa 2689 2492}%
\special{pa 2698 2481}%
\special{pa 2706 2470}%
\special{pa 2715 2460}%
\special{pa 2723 2449}%
\special{pa 2732 2438}%
\special{pa 2740 2426}%
\special{pa 2756 2404}%
\special{pa 2764 2392}%
\special{pa 2771 2381}%
\special{pa 2779 2369}%
\special{pa 2786 2358}%
\special{pa 2821 2298}%
\special{pa 2827 2285}%
\special{pa 2839 2261}%
\special{pa 2845 2248}%
\special{pa 2851 2236}%
\special{pa 2857 2223}%
\special{pa 2862 2210}%
\special{pa 2868 2198}%
\special{pa 2883 2159}%
\special{pa 2887 2146}%
\special{pa 2892 2133}%
\special{pa 2900 2107}%
\special{pa 2904 2093}%
\special{pa 2912 2067}%
\special{pa 2915 2053}%
\special{pa 2919 2040}%
\special{pa 2922 2027}%
\special{pa 2925 2013}%
\special{pa 2928 2000}%
\special{pa 2931 1986}%
\special{pa 2933 1972}%
\special{pa 2935 1959}%
\special{pa 2938 1945}%
\special{pa 2940 1931}%
\special{pa 2941 1918}%
\special{pa 2943 1904}%
\special{pa 2944 1890}%
\special{pa 2946 1876}%
\special{pa 2947 1863}%
\special{pa 2949 1835}%
\special{pa 2949 1821}%
\special{pa 2950 1808}%
\special{pa 2950 1780}%
\special{fp}%
% VECTOR 2 0 3 0 Black White  
% 2 530 2230 3180 1340
% 
\special{pn 8}%
\special{pa 530 2230}%
\special{pa 3180 1340}%
\special{fp}%
\special{sh 1}%
\special{pa 3180 1340}%
\special{pa 3110 1342}%
\special{pa 3129 1357}%
\special{pa 3123 1380}%
\special{pa 3180 1340}%
\special{fp}%
% ELLIPSE 2 0 3 0 Black White  
% 4 1850 1780 2940 1940 2940 1940 2940 1940
% 
\special{pn 8}%
\special{ar 1850 1780 1090 160 0.0000000 6.2831853}%
\end{picture}}%
}
}
\end{minipage}
&
\begin{minipage}[c]{0.15\textwidth}
\hspace{-12pt}
\uncover<4->{\textcolor{red}{$\mathbb{S}^n_s$と$\mathbb{H}^n_{n-s}$は反等長}}
\hspace{-3pt}
\end{minipage}
&
\begin{minipage}[c]{0.4\textwidth}
\hspace{-12pt}
\uncover<3->{
\scalebox{0.4}{\input{pic_pseudo-sphere-8}}
}
\end{minipage}
\end{tabular}
\end{table}
}





\frame{
\frametitle{\insertsection}
  \begin{table}
   \hspace{-10pt}
    \begin{tabular}{cl}
      1905 & {\bf 特殊相対性理論}(A. Einstein)\\
           & この世界を, 空間\(3\)次元と時間\(1\)次元の"時空"として考える\\
           & 正定値とは限らない計量を持つ多様体を導入\\
           & \\
      1908 & {\bf ミンコフスキー幾何学}(H. Minkowski)\\
           & 特殊相対性理論を幾何学として再構成\\
           & 時空は\(4\)次元の空間として記述される \\
           & \\
      1915-1916& {\bf 一般相対性理論}(A. Einstein)\\
           & 重力を時空の曲がりとして捉える, リーマン幾何学を応用        
    \end{tabular}
  \end{table}
  \begin{center}
    \raisebox{-3mm}[0mm]{\includegraphics[width=15pt, height=20pt, scale=0.2]{yazirusitate.png}} 
    \vspace{7pt}

    {\bf 擬リーマン幾何学}
    \end{center}
}



% \begin{frame}{\insertsection}
% \underline{例2.} \quad 重力場における光の軌跡
% \vspace{-5pt}
% \begin{center}
% \scalebox{0.25}{\includegraphics{logoRikadai.png}}   
% \end{center}

% {\bf \underline{aの位置にある天体が,b地点から見える(重力レンズ効果)}}

% \begin{itemize}
%       \item 光は常に“真っ直ぐ”進む
%       \item 光の粒子の軌跡はnull測地線
% \end{itemize}
% \end{frame}



%   \begin{frame}[noframenumbering]{\insertsection}
%     \underline{例2.} \quad 重力場における光の軌跡
%     \vspace{-5pt}
%     \begin{center}
%     \scalebox{0.13}{\includegraphics{logoRikadai3.jpg}}     
%     \end{center}
    
%     {\bf \underline{aの位置にある天体が,b地点から見える(重力レンズ効果)}}
    
%     \begin{itemize}
%           \item 光は常に“真っ直ぐ”進む
%           \item 光の粒子の軌跡はnull測地線
%     \end{itemize}
%     \end{frame}


\section{本日の発表の流れ}


\frame{
  \frametitle{\insertsection}
$(\, \tilde{M}, \langle \,,\rangle \,)$ : 擬球面 $\mathbb{S}^m_s$ \,または\, 擬双曲空間
$\mathbb{H}^m_s$ \vspace{5pt} 
\\
$(\, M, \langle \,, \rangle \,)$ : $\tilde{M}$の擬リーマン曲面,
\\
\hspace{45pt}
% \uncover<2->{
\textcolor<1-5>{red}{形作用素が対角化不可能, 平均曲率とスカラー曲率が一定}
  % }
\vspace{3pt} 
\\
\hspace{15pt} $\gamma$ \hspace{16pt} : $\tilde{M}$のnull曲線 \quad 
i.e. \(\langle \dot{\gamma}, \dot{\gamma}\rangle=0\)\, かつ\, \(\dot{\gamma} \neq 0\)
\vspace{20pt}
\pause
\\
{\large \underline{\, Part 1\, \(\cdots \) 擬双曲的ガウス写像による分類}}\vspace{2pt}
\begin{itemize}
\item \textcolor<6->{red}{B-scrollまたはcomplex circleの擬双曲的ガウス写像}
%\item 平行曲面の擬双曲的ガウス写像
% \item B-scrollの一般次元版であるgeneralized umbilical hypersurfaceの\\擬双曲的ガウス写像  
\vspace{10pt}
\end{itemize}
\pause

{\large \underline{\, Part 2 \, \(\cdots \) generalizations of B-scroll in \(\tilde{M}^{m}_s\)}}\vspace{2pt}
\begin{itemize}
\item generalized umbilical hypersurface in \(\tilde{M}^{n+1}_1\) \, \(\cdots\) 次元一般化 
\pause
\item \textcolor<6->{red}{generalized umbilical hypersurface in \(\tilde{M}^{n+1}_2\) \, \(\cdots\) 指数及び次元一般化} 
\pause
\item \textcolor<6->{red}{generalized B-scroll\, in \, \(\tilde{M}^5_2\) \, \(\cdots\) 指数及び余次元一般化}
\end{itemize}
}




\section{Part 1-1 \quad 擬双曲的ガウス写像による分類}




% \frame{
%   \begin{center}
%     {\Large {\bf Part \Rnum{1}}}
%   \end{center}
% \begin{beamercolorbox}[
%   %wd=60mm, sep=2pt,
%   %shadow=true, 
%   rounded=true
%   ]{frametitle}
%   \begin{center}
%     {\Large 擬双曲的ガウス写像による分類}
%   \end{center}
%   \end{beamercolorbox}
% }



\subsection{研究背景}


\frame{
\frametitle{\insertsection \quad \quad{\normalsize $\cdots$ \insertsubsection}}
\({\bf x} : M \hookrightarrow \mathbb{S}^m\) : 等長はめ込み
\vspace{3pt}

\hspace{3pt}\((e_1^p, \ldots, e_n^p)\)\hspace{3pt} : \(M\)の向きと適合する\(T_pM\)の正規直交フレーム
\begin{table}
 \begin{tabular}{lll}
   ガウス写像\(\nu\)
 & \(\overset{\mathrm{def}}{\Leftrightarrow}\)
 & \(\nu(p) := e_1^p \wedge \cdots \wedge e_n^p\)\\
 球面的ガウス写像\(\tilde{\nu}\)
 & \(\overset{\mathrm{def}}{\Leftrightarrow}\)
 & \(\tilde{\nu}(p) := \textcolor{red}{{\bf x}(p)} \wedge e_1^p \wedge \cdots \wedge e_n^p \)
 \end{tabular}
 \end{table}
 \vspace{10pt}
 \pause

$\phi: M\rightarrow \mathbb{S}^{m}_s\subset \mathbb{E}^{m+1}_s$ (or $ \mathbb{H}^{m}_{s}\subset \mathbb{E}^{m+1}_{s+1}$) : $C^\infty$級写像\\
\begin{table}
\begin{center}
\begin{tabular}{lll}
$\phi$ : \(k\)-type & $\overset{\mathrm{def}}{\Leftrightarrow}$ & $\phi=\phi_1+ \phi_2 +\cdots +\phi_k$,\quad $\Delta \phi_i=\lambda_i\phi_i$ \, ($\lambda_i\in \mathbb{R}$)
\end{tabular}
\end{center}
\end{table}
\vspace{10pt}
\pause

\underline{部分多様体の平均曲率ベクトル場\(H\)と部分多様体のtype numberの関係}
\begin{table}
  \hspace{-10pt}
   \begin{tabular}{cl}
     1970年代 & ・\(\Delta H=\lambda H\)となるリーマン部分多様体のtype number\\
          & ・null \(2\)-typeかつ \(H\) : const\, なローレンツ曲面の完全分類\vspace{5pt}\\
     1980年代 & ・部分多様体のガウス写像のtype number \vspace{5pt}\\
     2007年 & ・球面にはめ込まれた部分多様体の球面的ガウス写像   
   \end{tabular}
   \vspace{10pt}
 \end{table}
}



% \frame{
%   \frametitle{\insertsection \quad \quad{\normalsize $\cdots$ \insertsubsection}}
%   \underline{擬双曲的ガウス写像}
% \begin{center}
% \scalebox{0.8}{\input{pic_pseudo-hyperbolicGaussmap_Mensetsu.tex}}
% \end{center}
% }


\subsection{in \(\mathbb{S}^3_1\) or \(\mathbb{H}^3_1\)}

\frame{
  \frametitle{\insertsection \quad \quad{\normalsize $\cdots$ \insertsubsection}}
%\begin{center}
\vspace{-5pt}
\hspace{0pt}
%\begin{minipage}[c]{0.9\textwidth}
%\scalebox{0.9}{
\begin{thm}[D. S. Kim--Y. H. Kim]
$M^2_1\subset \mathbb{S}^3_1$ (または$\mathbb{H}^3_1)$ : 向きづけられたローレンツ超曲面
\begin{center}
\textcolor<2->{red}{形作用素が対角化不可能 かつ\, $H$,$K$ が一定}\vspace{3pt}\\
\rotatebox{90}{{\large $\Leftrightarrow$}} \vspace{-2pt}\\
$M^2_1$はB-scrollまたはcomplex circleどちらかの開部分 \vspace{-0pt}
\end{center}
\end{thm}
%}
%\end{minipage}
%\end{center}
\pause
\begin{center}
\raisebox{-2mm}[0mm]{\includegraphics[width=15pt, height=20pt, scale=0.2]{yazirusitate.png}} 
\vspace{-8pt}
\end{center}
\begin{table}
\scalebox{0.95}{
\begin{tabular}{|l||l|c|c|c|c|} \hline 
& \multicolumn{2}{c|}{\raisebox{-1mm}{\shortstack{\rule{0pt}{0.5pt}\\ $M^2_1$\\ \rule{0pt}{0.5pt}}}} & \raisebox{1mm}{$\tilde{\nu}$} & \raisebox{1mm}{$K$} & \raisebox{1mm}{$H$} \vspace{0pt} \\ \hline \hline %\cline{2-6} %\cline{1-6} 
& \uncover<3->{\raisebox{-0mm}{\shortstack{\rule{0pt}{1pt}\\ \textcolor<3->{red}{(Main Result 1)} \,\, $S^1_{\mathbb{C}}(\kappa)$}}} & \uncover<3->{\raisebox{-0mm}{$\kappa =-1$}} & \uncover<3->{\raisebox{-0mm}{$1$-type}} & \uncover<3->{\raisebox{-0mm}{$0$}} & \uncover<3->{\raisebox{-0mm}{$0$}} \\ \cline{3-6}
\uncover<3->{in \,$\mathbb{H}^3_1$} & \uncover<3->{\hspace{15pt}complex circle} & \uncover<3->{\raisebox{0mm}{$\kappa \neq -1$}} & \uncover<3->{$\infty$-type} & \uncover<3->{\raisebox{-0mm}{$0$}} & \uncover<3->{\raisebox{-0mm}{$\neq0$}} \\ \cline{2-6}
& \uncover<3->{\raisebox{-0mm}{\shortstack{\rule{0pt}{1pt}\\ \textcolor<3->{red}{(Main Result 2)} \,\, $\mathcal{B}(k_2)$}}} & \uncover<3->{$k_2=\pm1$} & \uncover<3->{\raisebox{-0mm}{$\infty$-type}} & \uncover<3->{\raisebox{-0mm}{$0$}} & \uncover<3->{\raisebox{-0mm}{$\neq0$}} \\ \cline{3-6}
& \uncover<3->{\raisebox{-0mm}{\hspace{30pt}B-scroll}} & \uncover<3->{$k_2 \neq \pm1$} & \uncover<3->{null $2$-type} & \uncover<3->{\raisebox{-0mm}{$\neq0$}} & \uncover<3->{\raisebox{-0mm}{$\neq0$}} \\ \cline{1-6} 
\raisebox{2mm}{in \,$\mathbb{S}^3_1$}\vspace{0pt} & \raisebox{-0mm}{\shortstack{\rule{0pt}{1pt}\\ (B--C--D 2017) \hspace{7pt} $\mathcal{B}(k_2)$\\ \hspace{-8pt}B-scroll}}  & & \raisebox{2.5mm}{null $2$-type}\vspace{-0pt} & \raisebox{2.5mm}{$\neq 0$} & \raisebox{2.5mm}{$\neq0$} \\ \hline
%& \raisebox{0mm}{\textcolor<7->{red}{B-scroll}} & & & & \\ \hline
\end{tabular}
}
\end{table}
}




\section{Part 1-2 \quad B-scroll}

\subsection{定義}

\frame{
  \frametitle{\insertsection \quad \quad{\normalsize $\cdots$ \insertsubsection}}

    \hspace{18pt}\(\gamma\) \hspace{14pt}: \(\mathbb{S}^{3}_1\)または\(\mathbb{H}^{3}_1\)のnull曲線

    \((A, B, C)\) : \(\gamma\)上の\textcolor{red}{Cartan frame field}
    \begin{center}
    i.e. \,
    \(
    \left\{
    \begin{array}{l}
    \langle A,A\rangle = \langle B,B\rangle=0,\quad \langle A,B\rangle=-1, \\
    \langle A,C\rangle = \langle B,C\rangle=0,\quad \langle C,C\rangle=1,\\
    \dot{\gamma}(s)=A(s), \\
    \dot{A}(s)=k_1(s)C(s), \\
    \dot{C}(s)=\textcolor<3->{red}{k_2(s)} A(s)+k_1(s)B(s),\\
    \dot{B}(s)=\textcolor<3->{red}{k_2(s)} C(s)+\varepsilon \gamma(s). 
    \end{array}
    \right.
    \)
    \end{center}
    \vspace{-6pt}
    \pause
  \begin{defi}
    \(M\)を,次のようにパラメータづけされたローレンツ曲面とする:
    \vspace{-7pt}
    \begin{table}
    \begin{tabular}{lll}
    \({\bf x} : M \hookrightarrow \mathbb{S}^{3}_1 \,\text{or}\, \mathbb{H}^{3}_1\)
    & \(\overset{\mathrm{def}}{\Leftrightarrow}\)
    & \({\bf x}(s, t) := \gamma(s)+tB(s)\)
    \vspace{-7pt}
    \end{tabular}
    \end{table}
    このとき,\vspace{-10pt}
    \begin{table}
      \begin{tabular}{lll}
      \(M\) : \textcolor{red}{{\it \(\gamma\)上のB-scroll}}
      & \(\overset{\mathrm{def}}{\Leftrightarrow}\)
      & \textcolor<3->{red}{\(k_2\)} : const
      \end{tabular}
      \end{table}
      \vspace{-8pt}
  \end{defi}
}



\frame{
  \frametitle{\insertsection \quad \quad{\normalsize $\cdots$ \insertsubsection}}
  \begin{table}
  \begin{tabular}{lr}
    \hspace{-50pt}
  \begin{minipage}{0.35\textwidth}
\scalebox{0.7}{\input{pic_B-scroll-10}}
  \end{minipage}
  \hspace{30pt}
&
\begin{minipage}{0.65\textwidth}
  \underline{Note}
\begin{itemize}
\item null曲線\(\gamma\)と\(\gamma\)に沿うFrenet型フレーム場から構成される線織面
\item \(2\)次元非退化ローレンツ曲面
\item 形作用素は対角化不可能,実固有値
\end{itemize}
\end{minipage}
\end{tabular}
\end{table}
}








\section{Part 2-1 \quad B-scrollの次元一般化}

% \frame{
%   \begin{center}
%     {\Large {\bf Part \Rnum{2}}}
%   \end{center}
% \begin{beamercolorbox}[
%   %wd=60mm, sep=2pt,
%   %shadow=true, 
%   rounded=true
%   ]{frametitle}
%   \begin{center}
%     {\Large generalizations of B-scroll in \(\tilde{M}^{m}_s\)}
%   \end{center}
%   \end{beamercolorbox}
% }

\subsection{定義}







\frame{
\frametitle{\insertsection \quad{\normalsize $\cdots$ \insertsubsection}}
\(\tilde{M}^{n+1}_1 =\) \textcolor{red}{\(\mathbb{E}^{n+1}_1\) or \(\mathbb{S}^{n+1}_1\) or \(\mathbb{H}^{n+1}_1\)}
\vspace{3pt}

\(M^n_1\) \hspace{9pt}: \(\tilde{M}^{n+1}_1\)のローレンツ超曲面
\vspace{3pt}

\hspace{3pt}\(A\) \hspace{14pt}: \(M^n_1\)の形作用素, \textcolor<4->{red}{対角化不可能, 実固有値}
\vspace{3pt}

\(P(x)\) \hspace{4pt}: \(A\)の最小多項式
%\vspace{10pt}
\begin{thm}[M. A. Magid]
\(A\)の固有値の個数が1 or 2個 かつ \(0\)でない固有値が1つ以下
\vspace{8pt}

\hspace{100pt}
\rotatebox{90}{{\Large \(\Leftarrow\)}}
\vspace{8pt}

\(M^n_1\)は次のいずれか:\vspace{8pt}
\begin{enumerate}
  \item \(M^n_1\) : generalized cylinder \uncover<2->{\quad \(\cdots\) ゼロ固有値を持つ}
  \item \(M^n_1\) : generalized umbilical hypersurface \uncover<2->{\quad \(\cdots\) ゼロ固有値を持たない}
\end{enumerate}
\vspace{-0pt}
\uncover<3->{
\hspace{60pt}
  \(
    \left\{
  \begin{aligned}
    & \, \text{degree} \, 2 \textcolor<4->{red}{\quad P(x)=(x-a)^2} \uncover<4->{\quad \textcolor{red}{\leftarrow \text{B-scrollはこの型}}}\\
    & \, \text{degree} \, 3 \quad P(x)=(x-a)^3
  \end{aligned}
    \right.
  \)
    }
\end{thm}
}


\subsection{形作用素}


\frame{
  \frametitle{\insertsection}
\(\tilde{M}\)が指数\(1\)のとき, 形作用素\(A\)はある基底に関して次の\(4\)つのいずれかの形をとる.
\[
  \begin{aligned}
  &\mbox{(I)}
  \,\,
  \begin{pmatrix}
  a_1 &       & 0 \\
      &\ddots &   \\
  0   &       & a_n    
  \end{pmatrix}
  ,
  &&
  \textcolor{red}{
  \mbox{(II)}
  \,\,
  \begin{pmatrix}
  a_0 & 0   & & & \\
  1   & a_0 & & & \hspace{-15pt}  \\
      &     & a_1 & & \\
      &  & & \ddots & \\
      &          & &        & a_{n-2}
  \end{pmatrix}
  ,
  }
  \\
  &\mbox{(III)}
  \,\,
  \begin{pmatrix}
  a_0 & 0   & 0   & & & \\
  0   & a_0 & 1   & & & \\
  -1  & 0   & a_0 & & & \\
      &     &     & a_1 & & \\
      &     &     &     & \ddots & \\
      &     &     &     &        & a_{n-3}
  \end{pmatrix}
  ,
  &&\mbox{(IV)}
  \,\,
  \begin{pmatrix}
  a_0 & b_0   & & & \\
  -b_0   & a_0 & & & \hspace{-15pt}  \\
      &     & a_1 & & \\
      &  & & \ddots & \\
      &          & &        & a_{n-2}
  \end{pmatrix}
  ,
  \end{aligned}
  \]
  ここで,$b_0\neq0$. 

}






\section{Part 2-2 \quad B-scrollの次元及び指数一般化}


\subsection{形作用素}


\frame{
  \frametitle{\insertsection \quad{\normalsize $\cdots$ \insertsubsection}}
\underline{\quad$\tilde{M}^m_2$の形作用素\quad} \\
\vspace{5pt}
{\footnotesize
\textcolor{red}{
$
\left(
{\renewcommand\arraystretch{1}
{\arraycolsep = 1mm
\begin{array}{cc|ccc}
\lambda & 0 & & & \\
 1 & \lambda &  & \\ \hline
 & &  \lambda &   & \\
 & &  & \ddots & \\
 & &    &   & \lambda 
\end{array}
}}
\right)
$
}
}
 ,\quad
{\footnotesize
$
\left(
{\renewcommand\arraystretch{0.5}
{\arraycolsep = 1mm
\begin{array}{cc|cc|ccc}
\lambda & 0 & & & & & \\
1 & \lambda & & & & & \\ \hline
 & & \lambda & 0 & & & \\
 & & 1 & \lambda & & & \\ \hline
 & &    &   & \lambda & & \\
 & &    &   &   & \ddots & \\
 & &    &   &   &        & \lambda
\end{array}
}}
\right)
$
}
 ,\quad
{\footnotesize
$
\left(
{\renewcommand\arraystretch{0.5}
{\arraycolsep = 1mm
\begin{array}{ccc|ccc}
\lambda &  &  &  & & \\
 1&  \lambda &  &  & & \\
 & 1& \lambda  &  & & \\ \hline
 &  &  & \lambda &  & \\
 &  &  &   & \ddots &  \\
 &  &  &   &   & \lambda  
\end{array}
}}
\right)
$
}
 ,\quad
\vspace{10pt} \\
{\footnotesize
$
\left(
{\renewcommand\arraystretch{0.8}
{\arraycolsep = 1mm
\begin{array}{ccc|ccc|ccc}
\lambda &         &         &   &   &   &   &        &   \\
1       & \lambda &         &   &   &   &   &        &   \\
        &  1      & \lambda &   &   &   &   &        &   \\ \hline
        &         &         & \lambda &         &  &   &        &   \\ 
        &         &         &   1     & \lambda &   &   &        &   \\
        &         &         &         &   1     & \lambda &   &        &   \\ \hline
 &   &   &   &   &   & \lambda &        &   \\
 &   &   &   &   &   &   & \ddots &   \\
 &   &   &   &   &   &   &        & \lambda \\
\end{array}
}}
\right)
$ ,\quad
{\footnotesize
$
\left(
{\renewcommand\arraystretch{0.5}
{\arraycolsep = 1mm
\begin{array}{cccc|ccc}
\lambda & 0 & & & & & \\
1 & \lambda & & & & & \\
 & 1 & \lambda & 0 & & & \\
 & & 1 & \lambda & & & \\ \hline
 & &    &   & \lambda & & \\
 & &    &   &   & \ddots & \\
 & &    &   &   &        & \lambda
\end{array}
}}
\right)
$
}
 ,\quad
$
\left(
{\renewcommand\arraystretch{0.5}
{\arraycolsep = 1mm
\begin{array}{ccccc|ccc}
\lambda&          &           &          &   &   &   &          \\
1      &  \lambda &           &          &   &          &          &   \\
       &      1   & \lambda   &          &   &   &   &          \\
       &          &  1        &  \lambda &   &  &   &          \\
       &          &           &    1     & \lambda &   &          &   \\ \hline
       &          &           &          &      & \lambda &        &   \\
       &          &           &          &   &     & \ddots &   \\
       &          &           &          &   &   &   &         \lambda  
\end{array}
}}
\right)
$
}
\pause
\begin{alertblock}{Main Result 3 (H. Kobayashi)}
  degree \(2\) のgeneralized umbilical hypersurface in \textcolor{red}{\(\mathbb{S}^{n+1}_2\) or \(\mathbb{H}^{n+1}_2\)} の具体例を構成
\end{alertblock}
}






% \frame{
% \frametitle{\insertsection \quad{\normalsize $\cdots$ \insertsubsection}}
% \(\tilde{M}^{n+1}_1\) : ローレンツ多様体,\(\mathrm{dim}\tilde{M}=n+1\)
% \vspace{3pt}

% \(M^n_1\) \hspace{9pt}: \(\tilde{M}^{n+1}_1\)のローレンツ超曲面,\(\mathrm{dim}M=n\)
% \vspace{3pt}

% \hspace{3pt}\(A\) \hspace{14pt}: \(M^n_1\)の形作用素, 対角化不可能
% \begin{defi}
%   \begin{table}
%     \begin{tabular}{lll}
%       \(M^n_1\) : \textcolor{red}{{\it generalized umbilical hypersurface}}
%       & \(\overset{\mathrm{def}}{\Leftrightarrow}\)
%       & \(A\)が\(0\)でない唯一の\\
%       &
%       &実固有値を持つ
%     \end{tabular}
%   \end{table}
% \(A\)の最小多項式が\(P(x)=(x-a)^2\)のとき,\(M^n_1\)を\textcolor{red}{degree \(2\)のgeneralized umbilical hypersurface}という (\(a\in \mathbb{R}\) : const).
% \end{defi}

% \pause
% \begin{alertblock}{Main Result 3 (H. Kobayashi--N. Koike)}
%       {\it
%       \(M^n_1\) : degree \(2\)の g. u. h. , \quad \(S\) : \(M^n_1\)のスカラー曲率
%       \vspace{-3pt}
%       \begin{table}
%       \begin{tabular}{lll}
%        $S=0$ \hspace{12pt} & $\Rightarrow$ &  \hspace{-10pt} $\tilde{\nu}$ : infinite type \\
%        $S\neq 0$(一定) & $\Rightarrow$& \hspace{-10pt} $\tilde{\nu}$ : null $2$-type
%       \end{tabular}
%       \end{table}
%       \vspace{-10pt}
%       }
%       \end{alertblock}
%}




% \frame{
%   \frametitle{\insertsection \quad {\normalsize $\cdots$ \insertsubsection}}
%   \begin{minipage}{0.9\textwidth}
%     Cartan frame (E. Cartan) : \(\mathbb{E}^3_1\)におけるnull曲線に沿う\\
%     \hspace{122pt} Frenet型フレーム
%     \vspace{-5pt}
%     \begin{flushleft}
%       \hspace{60pt}
%       \raisebox{-2mm}[0mm]{\includegraphics[width=15pt, height=20pt, scale=0.2]{yazirusitate.png}} 
%       \hspace{10pt}
%       {\footnotesize 一般次元ローレンツ多様体に拡張}
%       \vspace{-5pt}
%       \end{flushleft}
%     general Frenet frame (K. L. Duggal--A. Bejancu)
%     \vspace{-5pt}
%     \uncover<2->{
%     \begin{flushleft}
%       \hspace{60pt}
%       \raisebox{-2mm}[0mm]{\includegraphics[width=15pt, height=20pt, scale=0.2]{yazirusitate.png}} 
%       \hspace{10pt}
%       {\footnotesize よりシンプルな形に再構成}
%       \vspace{-5pt}
%       \end{flushleft}
%       natural Frenet frame (D. H. Jin)
%       \vspace{-5pt}
%       \begin{flushleft}
%         \hspace{60pt}
%         \raisebox{-2mm}[0mm]{\includegraphics[width=15pt, height=20pt, scale=0.2]{yazirusitate.png}} 
%         \hspace{10pt}
%         {\footnotesize 特殊なパラメータをとることでさらにシンプルに}
%         \vspace{-5pt}
%         \end{flushleft}
%       Cartan frame (A. Ferr\'{a}ndez--A. Gim\'{e}nez--P. Lucas)
%     }
%   \end{minipage}
%   \begin{minipage}{0.09\textwidth}
%     \uncover<5->{
%    \raisebox{-48mm}[0mm]{\includegraphics[height=130pt, scale=0.5]{ko-yazirushi.png}} 
%     }
%   \end{minipage}
%   \vspace{-2pt}
%   \uncover<3->{
%   \begin{itemize}
%     \item これらは構成の際にscreen vercor bundleなどの概念が必要
%   \vspace{7pt}
%   \end{itemize}
%   }
%   \uncover<4->{
%   \underline{一方,H. Kobayashi--N. Koikeによる構成方法は}
%   \begin{itemize}
%     \item general Frenet frameを経由せず構成
%     \item 高度な概念必要なし
%     \item \(\gamma\)のパラメータに条件なし
%   \end{itemize}
%   }
% }










% \frame{
%   \frametitle{\insertsection \quad{\normalsize $\cdots$ \insertsubsection}}
%   \underline{\(\tilde{M}^n_1\)の形作用素} \vspace{5pt}

% \(\tilde{M}\)が指数\(1\)のとき, \(A_N\)はある基底に関して次の\(4\)つのいずれかの形を

% とる( $b_0\neq0$ ).
% \vspace{-5pt}
% \[
%   \begin{aligned}
%   &\mbox{(I)}
%   \,\,
%   \begin{pmatrix}
%   a_1 &       & 0 \\
%       &\ddots &   \\
%   0   &       & a_n    
%   \end{pmatrix}
%   ,
%   &&
%   \textcolor{red}{
%   \mbox{(II)}
%   \,\,
%   \begin{pmatrix}
%   a_0 & 0   & & & \\
%   1   & a_0 & & & \hspace{-15pt}  \\
%       &     & a_1 & & \\
%       &  & & \ddots & \\
%       &          & &        & a_{n-2}
%   \end{pmatrix}
%   ,}
%   \\
%   &\mbox{(III)}
%   \,\,
%   \begin{pmatrix}
%   a_0 & 0   & 0   & & & \\
%   0   & a_0 & 1   & & & \\
%   -1  & 0   & a_0 & & & \\
%       &     &     & a_1 & & \\
%       &     &     &     & \ddots & \\
%       &     &     &     &        & a_{n-3}
%   \end{pmatrix}
%   ,
%   &&\mbox{(IV)}
%   \,\,
%   \begin{pmatrix}
%   a_0 & b_0   & & & \\
%   -b_0   & a_0 & & & \hspace{-15pt}  \\
%       &     & a_1 & & \\
%       &  & & \ddots & \\
%       &          & &        & a_{n-2}
%   \end{pmatrix}
%   ,
%   \end{aligned}
%   \]
% }








\subsection{Frenet型フレーム場}





% \frame{
%   \frametitle{\insertsection \quad{\normalsize $\cdots$ \insertsubsection}}
% Cartan frame (E. Cartan) : \(\mathbb{E}^3_1\)におけるnull曲線に沿う\\
% \hspace{122pt} Frenet型フレーム
% \vspace{-3pt}
% \begin{flushleft}
%   \hspace{60pt}
%   \raisebox{-2mm}[0mm]{\includegraphics[width=15pt, height=18pt, scale=0.2]{yazirusitate.png}} 
%   \hspace{10pt}
%   {\small 一般次元ローレンツ多様体\(M^n_1\)に拡張}
%   \vspace{-2pt}
%   \end{flushleft}
% general Frenet frame (K. L. Duggal--A. Bejancu)
% \vspace{-3pt}
% \begin{flushleft}
%   \hspace{60pt}
%   \raisebox{-2mm}[0mm]{\includegraphics[width=15pt, height=18pt, scale=0.2]{yazirusitate.png}} 
%   \hspace{10pt}
%   {\small よりシンプルな形に再構成}
%   \vspace{-2pt}
%   \end{flushleft}
%   natural Frenet frame (D. H. Jin)
%   \pause
%   \vspace{-3pt}
%   \begin{flushleft}
%     \hspace{60pt}
%     \raisebox{-2mm}[0mm]{\includegraphics[width=15pt, height=18pt, scale=0.2]{yazirusitate.png}} 
%     \hspace{10pt}
%     {\small 指数\(2\)に拡張}
%     \vspace{-2pt}
%     \end{flushleft}
%   natural Frenet frame with index \(2\) (K. L. Duggal--A. Bejancu--D. H. Jin)
%   \vspace{15pt}
%    \pause

%   \underline{別証明}
%   \vspace{-5pt}
%   \[
%     \left\{
%   \begin{aligned}
%     & \nabla_{\dot{\gamma}}A \text{\,: non-null かつ \(^\forall Z_i\) : non-nullのとき,Cartan frame (H. Kobayashi)}\\
%     & \nabla_{\dot{\gamma}}A \text{\,: nullのとき,bi-null Cartan frame (M. Sakaki--A. U\c{c}um--K. \.{I}larslan)}
%   \end{aligned}
%     \right.
%   \]
%   \pause
%   \vspace{-5pt}
%   \begin{alertblock}{Main Result 4 (H. Kobayashi)}
%   degree \(2\) のgeneralized umbilical hypersurface in \(\mathbb{S}^{n+1}_2\) or \(\mathbb{H}^{n+1}_2\) の具体例を構成
%   \end{alertblock}
% }








\section{Part 2-3 \quad B-scrollの指数及び余次元一般化}



\subsection{in \(\mathbb{E}^m_1\)}

\frame{
  \frametitle{\insertsection \quad \quad{\normalsize $\cdots$ \insertsubsection}}
  \vspace{3pt}

\((A, B, C, Z_1, \ldots, Z_{m-3})\) : \(\gamma\)に沿うCartan frame field
\vspace{-3pt}
\begin{table}
  \begin{tabular}{cll}
    \(M\) : \textcolor{red}{generalized B-scroll}
    & \(\overset{\mathrm{def}}{\Leftrightarrow}\)
    & \(
      M \,:\, {\bf x}(s, t)=\gamma(s)+tB(s)  
      \)
  \end{tabular}
\end{table}
\vspace{0pt}
\pause

\underline{Note}
\begin{itemize}
\item null曲線\(\gamma\)と\(\gamma\)に沿うFrenet型フレーム場から構成される線織面
\item \(2\)次元非退化ローレンツ曲面
\item 形作用素は対角化不可能,実固有値
\end{itemize}
\pause

\begin{theorem}[D. S. Kim--Y. H. Kim--D. W. Yoon]
  % \begin{table}
  % \begin{tabular}{ll}
  % \begin{minipage}{0.42\textwidth}
    \vspace{0pt}

\hspace{3pt}\(M\)\hspace{2pt} : \textcolor{red}{\(\mathbb{E}^m_1\)}のnull scroll \vspace{3pt}

\hspace{3pt}\(H\)\hspace{2pt} : \(M\)の平均曲率ベクトル場\vspace{3pt}

\(A_H\) : \(M\)の\(H\)方向の形作用素
  % \end{minipage}
  % &
  % \uncover<3->{
    
  % \begin{minipage}{0.49\textwidth}
  %   \color<3, 6>{red}{
  %     {\small ※\(M\)は\(2\)次元非退化ローレンツ線織面}
  %   \begin{screen}
  %     \vspace{-20pt}
  %      \[M :\, {\bf x}(s, t)=\gamma(s)+tB(s) \vspace{-10pt}\]
  %     %  {\small i.e. \(M\)は\(\gamma\)と\(\gamma\)に沿うFrenet型フレーム場から構成される線織面}
  %      \vspace{-15pt}
  %   \end{screen}

  %   \hspace{70pt} 
  %   \rotatebox{90}{\(\Leftrightarrow\)}
  %   {\rm {\footnotesize \raisebox{1mm}{def}}}
  %   }
  % \end{minipage}
  % }
  % \end{tabular}
  % \vspace{-20pt}
  % \end{table}

\begin{table}
  \begin{tabular}{cll}
    \(A_H\)の最小多項式が\((x-a^2)^2\)
    & \hspace{3pt}\(\Leftrightarrow\)
    & \(M\) : generalized B-scroll\\
    (\(a\in \mathbb{R}\) : const)
    &
    &
  \end{tabular}
  % \uncover<4->{
%   \vspace{-15pt}
% \end{table}
% \begin{table}
%   \begin{tabular}{cll}
%     \(A_H\)の最小多項式が\((x-\textcolor<5>{red}{k_2^2})^2\)
%     & \(\Leftarrow\)
%     & \(M\) : \textcolor<6>{red}{generalized B-scroll}
%   \end{tabular}
\end{table}
  %}
\vspace{-5pt}
\end{theorem}
% \uncover<3->{
%   \underline{\(\gamma \subset \mathbb{E}^m_1\)に沿うCartan frame field}
%   \vspace{-5pt}
% \begin{center}
% \begin{minipage}{0.7\textwidth}
%   % \underline{\(\gamma\) in \(\mathbb{E}^m_1\)}
%   {\small
%   {\renewcommand\arraystretch{0.3}
%   {\arraycolsep = 0.5mm
%   %\hspace{-30pt}
%   \(
%   \begin{array}{c}
%   A \\ B\\ C\\ Z_1\\ \vdots \vspace{2pt}\\ Z_m
%   \end{array}
%   \)
%   }
%   {\renewcommand\arraystretch{0.6}
%   {\arraycolsep = 0.8mm
%   \(
%   \left(
%   \begin{array}{ccccc}
%   0 & -1&   &   &   \\
%   -1& 0 &   &   &   \\
%     &   & 1 &   &   \\
%     &   &   & \ddots &  \\
%     &   &   &   & 1 
%   \end{array}
%   \right)
%   \)
%   }}
%   } 
%   \, and\quad
%   %\vspace{-15pt}
%   \(
%   \left \{
%   \begin{array}{l}
%   \dot{A}=k_1C\\
%   \dot{C}=\textcolor<5>{red}{k_2}A+k_1B\\
%   \dot{B}=\textcolor<5>{red}{k_2} C+Z_1 \\
%   \dot{Z}_1=k_3A+k_4Z_2 \\
%   \dot{Z}_2=-k_4Z_1+k_5Z_3 \vspace{-3pt}\\
%     \hspace{15pt}{\footnotesize \vdots}       
%   \end{array}
%   \right.
%   \)
%   }
%   \end{minipage}
% \end{center}
%}
}





% \frame{
%   \frametitle{\insertsection \quad \quad{\normalsize $\cdots$ \insertsubsection}}
% \uncover<1->{
% {\small $\gamma$ : $\tilde{M}^5_2$のnull曲線, \, $A:=\dot{\gamma}$.} \vspace{0pt} 

% \hspace{-225pt}
% \scalebox{0.8}{\input{pic_M^5_2Frenet-2_Mensetsu}}
% }

% \uncover<2->{
% \begin{table}
% \begin{tabular}{c|c}
% \begin{minipage}{0.42\textwidth}
% %\underline{{\footnotesize {\bf Case 1} or {\bf Case 2} }}
% \vspace{-3pt}
% \begin{block}{{\small Type (\rnum{1})\, (H. Kobayashi)}}
%   \vspace{-5pt}
% {\footnotesize
% \begin{eqnarray}\label{joken5}
% {\renewcommand\arraystretch{0.3}
% {\arraycolsep = 0.5mm
% \hspace{-30pt}
% \begin{array}{l}
% A \\ B\\ C\\ Z_1\\ Z_2
% \end{array}
% \,
% \left(
% \begin{array}{ccccc}
% 0 & -1&   &   &   \\
% -1& 0 &   &   &   \\
%   &   & \varepsilon_C &            & \\
%   &   &            & \varepsilon_1 & \\
%   &   &            &            & \varepsilon_2
% \end{array}
% \right)
% }}
% \end{eqnarray}
% \vspace{-15pt}
% \begin{eqnarray}\label{joken5-2}
% \left \{
% \begin{array}{l}
% \dot{A}=k_1C\\
% \dot{C}=k_2A+\varepsilon_C k_1B\\
% \dot{B}=\varepsilon_C k_2 C+k_3Z_1+\varepsilon \gamma \\
% \dot{Z}_1=\varepsilon_1 k_3A+k_4Z_2 \\
% \dot{Z}_2=\varepsilon_C k_4Z_1
% \end{array}
% \right.
% \end{eqnarray}
% }
% \vspace{-9pt}
% \end{block}
% \end{minipage}
% }
% \uncover<3->{
% &
% \hspace{3pt}
% \begin{minipage}{0.42\textwidth}
% %\underline{{\footnotesize {\bf Case 3} }}
% \vspace{-3pt}
% \begin{block}{{\small Type (\rnum{2})\, (H. Kobayashi)}}
%   \vspace{-5pt}
% {\footnotesize
% \begin{eqnarray}\label{joken6}
% {\renewcommand\arraystretch{0.3}
% {\arraycolsep = 0.5mm
% \hspace{-30pt}
% \begin{array}{l}
% A \\ B\\ C\\ Z_1\\ Z_2
% \end{array}
% \left(
% \begin{array}{ccccc}
% 0 & -1&   &   &   \\
% -1& 0 &   &   &   \\
%   &   & 1 &   &   \\
%   &   &   & 0 & -1\\
%   &   &   & -1& 0 
% \end{array}
% \right)
% }}
% \end{eqnarray}
% \vspace{-15pt}
% \begin{eqnarray}\label{joken6-2}
% \left \{
% \begin{array}{l}
% \dot{A}=k_1C\\
% \dot{C}=k_2A+k_1B\\
% \dot{B}=k_2 C+Z_1+\varepsilon \gamma \\
% \dot{Z}_1=hZ_1 \\
% \dot{Z}_2=-A-hZ_2
% \end{array}
% \right.
% \end{eqnarray}
% }
% \vspace{-9pt}
% \end{block}
% \end{minipage}
% }
% \end{tabular}
% \end{table}
% }


\subsection{in \(\mathbb{S}^5_2\) or \(\mathbb{H}^5_2\)}

\frame{
  \frametitle{\insertsection \quad{\normalsize $\cdots$ \insertsubsection}}
%   \begin{itemize}
%   \item $s=1$のとき \quad $\gamma$ : $\tilde{M}^n_1$のnull曲線, \, $A:=\dot{\gamma}$. 
%   \end{itemize}
% \medskip
% \hspace{-3pt}
% \scalebox{0.88}{\input{pic_M^n_1Frenet}}
% \vspace{28pt}\\
% \pause

null曲線 \(\gamma \subset \tilde{M}^m_2\)に沿うFrenet型フレーム場, \, 
\vspace{7pt}

$A:=\dot{\gamma}$ 
\medskip
\pause
\hspace{-265pt}
\scalebox{0.9}{\input{pic_M^n_2Frenet-4_Mensetsu}}
\vspace{5pt}
\pause

\uncover<3->{\hspace{200pt}{\small 別証明}}
\vspace{-8pt}

\hspace{30pt}
\(
    \left\{
  \begin{aligned}
    & \text{\, Type (\rnum {1}) \hspace{-0pt} : Cartan frame field \uncover<3->{\hspace{37pt} {\small \(\cdots\) H. Kobayashi}}}\\
    & \text{\, Type (\rnum {2}) : bi-null Cartan frame field \uncover<3->{\hspace{7pt} {\small \(\cdots\) M. Sakaki--A. U\c{c}um--K. \.{I}larslan}}}\\
    & \text{\, Type (\rnum {3}) \uncover<3->{\hspace{128pt} {\small \(\cdots \,m=5\)のとき,H. Kobayashi}}}
  \end{aligned}
    \right.
  \)
}











\subsection{in \(\mathbb{S}^5_2\) or \(\mathbb{H}^5_2\)}

\frame{
  \frametitle{\insertsection \quad \quad{\normalsize $\cdots$ \insertsubsection}}
  \vspace{-3pt}

\begin{alertblock}{Main Result 4 (H. Kobayashi)}
      {\it 
      % \hspace{30pt}
      % \(\gamma\) \hspace{29pt}: \textcolor{red}{\(\mathbb{S}^5_2\) or \(\mathbb{H}^5_2\)}のnull曲線
      % \vspace{3pt}

      % \((A, B, C, Z_1, Z_2)\) : $\gamma$に沿う\( \mathbb{S}^5_2\) or \( \mathbb{H}^5_2\)のCartan frame field \, s.t.\, \(k_2\) : const
      % \vspace{10pt}
      % \pause

      % \(M\)が次の条件を満たすとする:\vspace{-5pt}
      % \[
      %   (*)\quad
      %   \left\{
      % \begin{aligned}
      %   & M \text{\,: \(\gamma\)と\(\gamma\)に沿うFrenet型フレーム場から構成されるnull scroll}\\
      %   & M \text{\,: \(2\)次元非退化ローレンツ線織面}
      % \end{aligned}
      %   \right.
      %   \vspace{-10pt}
      % \]
      % % \[
      % % \text{s.t.} \,
      % % \left\{  
      % % \begin{array}{l}
      % % B \text{\,: nullベクトル場,}\\
      % % \langle A, B \rangle =-1 ,\,\, \langle B, C \rangle =0.
      % % \end{array}
      % % \right.
      % \pause
      % \begin{center}
      % \(\Rightarrow\)\, \(M\)は\, \({\bf x}(s, t)=\gamma(s)+tB(s)\)\, によりパラメータづけされる.
      % \end{center}
      % \vspace{-0pt}
      % % \[
      % %   \Rightarrow \quad
      % % M: \, {\bf x}(s, t):=\gamma(s)+tB(s)
      % %   \vspace{-0pt}
      % % \]
      % % \hspace{3pt} \(\tilde{H}\) \hspace{5pt}: \(M\)の平均曲率ベクトル場,
      % % \(H:=\tilde{H}/\Vert \tilde{H}\Vert\)
      % % \vspace{3pt}
      % \pause

    % \hspace{43pt}
    % \begin{table}
    % \begin{tabular}{ll}
    %   \begin{minipage}{0.45\textwidth}
    %{\small
    \hspace{6pt}\(M\)\hspace{5pt} : \, \textcolor{red}{\(\mathbb{S}^5_2\) or \(\mathbb{H}^5_2\)}におけるnull scroll \vspace{3pt}

    \hspace{6pt}\(H\)\hspace{5pt} : \, \(M\)の平均曲率ベクトル場\vspace{3pt}
    
    \hspace{4pt}\(A_H\)\hspace{2pt} : \, \(M\)の\(H\)方向の形作用素\vspace{3pt}

    \(P(x)\) : \, \(A_H\)の最小多項式\vspace{7pt}
   % }
  %     \end{minipage}
  %     &
  %     \begin{minipage}{0.5\textwidth}
  %       {\small 
  %       \begin{itembox}{Note}
  %     \begin{itemize}
  %     \item null曲線\(\gamma\)と\(\gamma\)に沿うFrenet型フレーム場から構成される線織面
  %     \item \(2\)次元非退化ローレンツ曲面
  %     \item 形作用素は対角化不可能,実固有値
  %     \end{itemize}
  %   \end{itembox}
  %       }
  %     \end{minipage}
  %   \end{tabular}
  % \end{table}

    \(M\)がgeneralized B-scrollのとき,\(P(x)\)は次のいずれかである:
    
    \vspace{-3pt}
    \begin{center}
      \begin{minipage}{0.7\textwidth}
    \begin{enumerate}
    \uncover<2->{\item \(Z_1\) : non-null  \hspace{5pt}$\Rightarrow$\hspace{5pt}  \( P(x)=(x-(\textcolor<4->{red}{\varepsilon_C k_2^2+\varepsilon_1 k_3^2}) )^2\)}  \\
    \uncover<3->{\item \(Z_1\) : null  \hspace{26pt}$\Rightarrow$ \hspace{2pt} \( P(x)=(x-\textcolor<4->{red}{k_2^2})^2\)}
    \end{enumerate}
  \end{minipage}
    \end{center}
      %\vspace{-2pt}
  }
  \vspace{-5pt}
\end{alertblock}
\vspace{-6pt}
\begin{table}
\begin{tabular}{cc}
\begin{minipage}{0.42\textwidth}
%\underline{{\footnotesize {\bf Case 1} or {\bf Case 2} }}
\vspace{-3pt}
\uncover<2->{
\begin{block}{{\small Type (\rnum{1})\, \(\cdots Z_1\)がnon-nullのとき}}
  \vspace{-8pt}
{\footnotesize
% \begin{eqnarray}\label{joken5}
% {\renewcommand\arraystretch{0.3}
% {\arraycolsep = 0.5mm
% \hspace{-30pt}
% \begin{array}{l}
% A \\ B\\ C\\ Z_1\\ Z_2
% \end{array}
% \,
% \left(
% \begin{array}{ccccc}
% 0 & -1&   &   &   \\
% -1& 0 &   &   &   \\
%   &   & \varepsilon_C &            & \\
%   &   &            & \varepsilon_1 & \\
%   &   &            &            & \varepsilon_2
% \end{array}
% \right)
% }}
% \end{eqnarray}
%\vspace{-15pt}
\begin{eqnarray}\label{joken5-2}
\left \{
\begin{array}{l}
\dot{A}=k_1C\\
\dot{C}=\textcolor<4->{red}{k_2}A+\varepsilon_C k_1B\\
\dot{B}=\textcolor<4->{red}{\varepsilon_C k_2} C+\textcolor<4->{red}{k_3}Z_1+\varepsilon \gamma \\
\dot{Z}_1=\textcolor<4->{red}{\varepsilon_1 k_3}A+k_4Z_2 \\
\dot{Z}_2=\varepsilon_C k_4Z_1
\end{array}
\right.
\end{eqnarray}
}
\vspace{-9pt}
\end{block}
}
\end{minipage}
&
\hspace{3pt}
\begin{minipage}{0.42\textwidth}
%\underline{{\footnotesize {\bf Case 3} }}
\vspace{-3pt}
\uncover<3->{
\begin{block}{{\small Type (\rnum{3})\, \(\cdots Z_1\)がnullのとき}}
  \vspace{-8pt}
{\footnotesize
% \begin{eqnarray}\label{joken6}
% {\renewcommand\arraystretch{0.3}
% {\arraycolsep = 0.5mm
% \hspace{-30pt}
% \begin{array}{l}
% A \\ B\\ C\\ Z_1\\ Z_2
% \end{array}
% \left(
% \begin{array}{ccccc}
% 0 & -1&   &   &   \\
% -1& 0 &   &   &   \\
%   &   & 1 &   &   \\
%   &   &   & 0 & -1\\
%   &   &   & -1& 0 
% \end{array}
% \right)
% }}
% \end{eqnarray}
%\vspace{-15pt}
\begin{eqnarray}\label{joken6-2}
\left \{
\begin{array}{l}
\dot{A}=k_1C\\
\dot{C}=\textcolor<4->{red}{k_2}A+k_1B\\
\dot{B}=\textcolor<4->{red}{k_2} C+Z_1+\varepsilon \gamma \\
\dot{Z}_1=hZ_1 \\
\dot{Z}_2=-A-hZ_2
\end{array}
\right.
\end{eqnarray}
}
\vspace{-9pt}
\end{block}
}
\end{minipage}
\end{tabular}
\end{table}
}




% \section{今後の課題}


% \frame{
%   \frametitle{\insertsection}
% \underline{\quad$\tilde{M}^n_2$の形作用素\quad} \\
% \vspace{10pt}
% {\footnotesize
% \textcolor{red}{
% $
% \left(
% {\renewcommand\arraystretch{1}
% {\arraycolsep = 1mm
% \begin{array}{cc|ccc}
% \lambda & 0 & & & \\
%  1 & \lambda &  & \\ \hline
%  & &  \lambda &   & \\
%  & &  & \ddots & \\
%  & &    &   & \lambda 
% \end{array}
% }}
% \right)
% $
% }
% }
%  ,\quad
% {\footnotesize
% $
% \left(
% {\renewcommand\arraystretch{0.5}
% {\arraycolsep = 1mm
% \begin{array}{cc|cc|ccc}
% \lambda & 0 & & & & & \\
% 1 & \lambda & & & & & \\ \hline
%  & & \lambda & 0 & & & \\
%  & & 1 & \lambda & & & \\ \hline
%  & &    &   & \lambda & & \\
%  & &    &   &   & \ddots & \\
%  & &    &   &   &        & \lambda
% \end{array}
% }}
% \right)
% $
% }
%  ,\quad
% {\footnotesize
% $
% \left(
% {\renewcommand\arraystretch{0.5}
% {\arraycolsep = 1mm
% \begin{array}{ccc|ccc}
% \lambda &  &  &  & & \\
%  1&  \lambda &  &  & & \\
%  & 1& \lambda  &  & & \\ \hline
%  &  &  & \lambda &  & \\
%  &  &  &   & \ddots &  \\
%  &  &  &   &   & \lambda  
% \end{array}
% }}
% \right)
% $
% }
%  ,\quad
% \vspace{10pt} \\
% {\footnotesize
% $
% \left(
% {\renewcommand\arraystretch{0.8}
% {\arraycolsep = 1mm
% \begin{array}{ccc|ccc|ccc}
% \lambda &         &         &   &   &   &   &        &   \\
% 1       & \lambda &         &   &   &   &   &        &   \\
%         &  1      & \lambda &   &   &   &   &        &   \\ \hline
%         &         &         & \lambda &         &  &   &        &   \\ 
%         &         &         &   1     & \lambda &   &   &        &   \\
%         &         &         &         &   1     & \lambda &   &        &   \\ \hline
%  &   &   &   &   &   & \lambda &        &   \\
%  &   &   &   &   &   &   & \ddots &   \\
%  &   &   &   &   &   &   &        & \lambda \\
% \end{array}
% }}
% \right)
% $ ,\quad
% {\footnotesize
% $
% \left(
% {\renewcommand\arraystretch{0.5}
% {\arraycolsep = 1mm
% \begin{array}{cccc|ccc}
% \lambda & 0 & & & & & \\
% 1 & \lambda & & & & & \\
%  & 1 & \lambda & 0 & & & \\
%  & & 1 & \lambda & & & \\ \hline
%  & &    &   & \lambda & & \\
%  & &    &   &   & \ddots & \\
%  & &    &   &   &        & \lambda
% \end{array}
% }}
% \right)
% $
% }
%  ,\quad
% $
% \left(
% {\renewcommand\arraystretch{0.5}
% {\arraycolsep = 1mm
% \begin{array}{ccccc|ccc}
% \lambda&          &           &          &   &   &   &          \\
% 1      &  \lambda &           &          &   &          &          &   \\
%        &      1   & \lambda   &          &   &   &   &          \\
%        &          &  1        &  \lambda &   &  &   &          \\
%        &          &           &    1     & \lambda &   &          &   \\ \hline
%        &          &           &          &      & \lambda &        &   \\
%        &          &           &          &   &     & \ddots &   \\
%        &          &           &          &   &   &   &         \lambda  
% \end{array}
% }}
% \right)
% $
% }
% }


\end{document}